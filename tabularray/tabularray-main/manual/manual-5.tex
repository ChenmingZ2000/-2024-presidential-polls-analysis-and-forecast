% -*- coding: utf-8 -*-
% !TEX program = lualatex
\documentclass[oneside]{book}

% -*- coding: utf-8 -*-
% !TEX program = lualatex

\newcommand*{\myversion}{2024A}
\newcommand*{\mylpad}[1]{\ifnum#1<10 0\the#1\else\the#1\fi}

\usepackage[a4paper,margin=2.5cm]{geometry}

\setlength{\parindent}{0pt}
\setlength{\parskip}{4pt plus 1pt minus 1pt}

\usepackage{codehigh} % https://ctan.org/pkg/codehigh
\usepackage{tabularray}
\usepackage{array,multirow,amsmath}
\usepackage{chemmacros,environ}
\usepackage{enumitem}

\usepackage[firstpage=true]{background}
\backgroundsetup{contents={}}

\UseTblrLibrary{
  amsmath,booktabs,counter,diagbox,functional,keyvalue,siunitx,varwidth
}

\usepackage{hyperref}
\hypersetup{
  colorlinks=true,
  urlcolor=blue3,
  linkcolor=blue3,
}

\usepackage{tcolorbox}
\tcbset{sharp corners, boxrule=0.5pt, colback=red9}

\usepackage{float}
%\usepackage{enumerate}

\setcounter{tocdepth}{1}

\newcommand*{\K}[1]{\texttt{#1}}
\newcommand*{\V}[1]{\texttt{#1}}
\newcommand*{\None}{$\times$}

\NewTblrEnviron{newtblr}
\SetTblrOuter[newtblr]{long}
\SetTblrInner[newtblr]{
  hlines = {white}, column{1,2} = {co=1}, colsep = 5pt,
  row{odd} = {brown8}, row{even} = {gray8},
  row{1} = {fg=white, bg=purple2, font=\bfseries\sffamily},
}

\NewTblrEnviron{spectblr}
\SetTblrOuter[spectblr]{long}
\SetTblrInner[spectblr]{
  hlines = {white}, column{2} = {co=1}, colsep = 5pt,
  row{odd} = {brown8}, row{even} = {gray8},
  row{1} = {fg=white, bg=purple2, font=\bfseries\sffamily},
  rowhead = 1,
}

\newcommand{\mywarning}[1]{%
  \begin{tcolorbox}
  The interfaces in this #1 should be seen as
  \textcolor{red3}{\bfseries experimental}
  and are likely to change in future releases, if necessary.
  Don’t use them in important documents.
  \end{tcolorbox}
}

%\renewcommand*{\thefootnote}{*}

\colorlet{highback}{azure9}
\CodeHigh{language=latex/table,style/main=highback,style/code=highback}
\NewCodeHighEnv{code}{style/main=gray9,style/code=gray9}
\NewCodeHighEnv{demo}{style/main=gray9,style/code=gray9,demo}

%\CodeHigh{lite}

\CodeHigh{lite}
\setcounter{chapter}{4}

\begin{document}

\chapter{Use Some Libraries}

The \verb!tabularray! package emulates or fixes some commands in other packages.
To avoid potential conflict, you need to enable them with \verb!\UseTblrLibrary! command.

\section{Library \texttt{amsmath}}

With \verb!\UseTblrLibrary{amsmath}! in the preamble of the document,
\verb!tabularray! will load \verb!amsmath! package, and define \verb!+array!, \verb!+matrix!,
\verb!+bmatrix!, \verb!+Bmatrix!, \verb!+pmatrix!, \verb!+vmatrix!, \verb!+Vmatrix! and \verb!+cases!
environments. Each of the environments is similar to the environment without \verb!+! prefix in its name,
but has default \verb!rowsep=2pt! just as \verb!tblr! environment. Every environment
except \verb!+array! accepts an optional argument, where you can write inner specifications.

\begin{demo}
$\begin{pmatrix}
 \dfrac{2}{3} &  \dfrac{2}{3} &  \dfrac{1}{3} \\
 \dfrac{2}{3} & -\dfrac{1}{3} & -\dfrac{2}{3} \\
 \dfrac{1}{3} & -\dfrac{2}{3} &  \dfrac{2}{3} \\
\end{pmatrix}$
\end{demo}

\begin{demohigh}
$\begin{+pmatrix}[cells={r},row{2}={purple8}]
 \dfrac{2}{3} &  \dfrac{2}{3} &  \dfrac{1}{3} \\
 \dfrac{2}{3} & -\dfrac{1}{3} & -\dfrac{2}{3} \\
 \dfrac{1}{3} & -\dfrac{2}{3} &  \dfrac{2}{3} \\
\end{+pmatrix}$
\end{demohigh}

\begin{demo}
$f(x)=\begin{cases}
 0,            & x=1; \\
 \dfrac{1}{3}, & x=2; \\
 \dfrac{2}{3}, & x=3; \\
 1,            & x=4. \\
\end{cases}$
\end{demo}

\begin{demohigh}
$f(x)=\begin{+cases}
 0,            & x=1; \\
 \dfrac{1}{3}, & x=2; \\
 \dfrac{2}{3}, & x=3; \\
 1,            & x=4. \\
\end{+cases}$
\end{demohigh}

\section{Library \texttt{booktabs}}

With \verb!\UseTblrLibrary{booktabs}! in the preamble of the document,
\verb!tabularray! will load \verb!booktabs! package,
and define
  \verb!\toprule!, \verb!\midrule!, \verb!\bottomrule!,
  \verb!\cmidrule!, \verb!\cmidrulemore!, \verb!\morecmidrules!,
  \verb!\specialrule!, \verb!\addrowspace!, and \verb!\addlinespace!
as table commands.

\begin{demohigh}
\begin{tblr}{llll}
\toprule
 Alpha   & Beta  & Gamma   & Delta \\
\midrule
 Epsilon & Zeta  & Eta     & Theta \\
\cmidrule{1-3}
 Iota    & Kappa & Lambda  & Mu    \\
\cmidrule{2-4}
 Nu      & Xi    & Omicron & Pi    \\
\bottomrule
\end{tblr}
\end{demohigh}

Just like \verb!\hline! and \verb!\cline! commands,
you can also specify rule width and color by using hline keys in the optional
argument of any of these commands.

Like in \verb!booktabs!, by default
  width of \verb!\toprule! and \verb!\bottomrule! are determined by \verb!\heavyrulewidth!,
  width of \verb!\midrule! is determined by \verb!\lightrulewidth!, and
  width of \verb!\cmidrule! and \verb!\cmidrulemore! are determined by \verb!\cmidrulewidth!, respectively.
All three \verb!\...rulewidth! are dimensions.

\begin{demohigh}
\begin{tblr}{llll}
\toprule[2pt,purple3]
 Alpha   & Beta  & Gamma  & Delta \\
\midrule[blue3]
 Epsilon & Zeta  & Eta    & Theta \\
\cmidrule[azure3]{2-3}
 Iota    & Kappa & Lambda & Mu    \\
\bottomrule[2pt,purple3]
\end{tblr}
\end{demohigh}

If you need more than one \verb!\cmidrule!s, you can use \verb!\cmidrulemore!
command, which is simpler than the \verb!booktabs! usage
\verb!\morecmidrules\cmidrule!.
\verb!\cmidrulemore! can receive hline keys in an optional argument too.

\begin{demohigh}
\begin{tblr}{llll}
\toprule
 Alpha   & Beta  & Gamma   & Delta \\
\cmidrule{1-3} \cmidrulemore{2-4}
 Epsilon & Zeta  & Eta     & Theta \\
\cmidrule{1-3} \morecmidrules \cmidrule{2-4}
 Iota    & Kappa & Lambda  & Mu    \\
\bottomrule
\end{tblr}
\end{demohigh}

From version 2021N (2021-09-01), you can set trimming positions of
\verb!\cmidrule! and \verb!\cmidrulemore!, using newly introduced trimming
options (\verb!leftpos!, \verb!rightpos!, \verb!endpos!, \verb!l!, \verb!r!,
and \verb!lr!) (see Section~\ref{sec:hlines-vlines}).
Option \verb!endpos! is already applied to these two commands.

\begin{demohigh}
\begin{tblr}{llll}
\toprule
 Alpha   & Beta  & Gamma   & Delta \\
\cmidrule[lr]{1-2} \cmidrule[lr=-0.4]{3-4}
 Epsilon & Zeta  & Eta     & Theta \\
\cmidrule[r]{1-2} \cmidrule[l]{3-4}
 Iota    & Kappa & Lambda  & Mu    \\
\bottomrule
\end{tblr}
\end{demohigh}

Since \verb!booktabs! tables usually don't have vlines, the meaningful values
here are decimal numbers between \verb!-1! and \verb!0!.
The default value \verb!-0.8! for \verb!l!, \verb!r!, and \verb!lr! is chosen to
make similar result as \verb!booktabs! package does.

There is also a \verb!booktabs! environment for you. With this environment,
the default \verb!rowsep=0pt!, but extra vertical space will be added by
\verb!\toprule!, \verb!\midrule!, \verb!\bottomrule! and \verb!\cmidrule! commands.
The sizes of vertical space are determined by \verb!\aboverulesep! and \verb!\belowrulesep! dimensions.

\begin{demohigh}
\begin{booktabs}{
  colspec = lcccc,
  cell{1}{1} = {r=2}{}, cell{1}{2,4} = {c=2}{},
}
\toprule
  Sample & I &   & II &   \\
\cmidrule[lr]{2-3} \cmidrule[lr]{4-5}
         & A & B & C & D \\
\midrule
  S1     & 5 & 6 & 7 & 8 \\
  S2     & 6 & 7 & 8 & 5 \\
  S3     & 7 & 8 & 5 & 6 \\
\bottomrule
\end{booktabs}
\end{demohigh}
% S4     & 8 & 5 & 6 & 7 \\

You can also use \verb!\specialrule! command.
The second argument sets \verb!belowsep! of previous row,
and the third argument sets \verb!abovesep! of current row,

\begin{demohigh}
\begin{booktabs}{row{2}={olive9}}
\toprule
 Alpha   & Beta  & Gamma   & Delta \\
\specialrule{0.5pt}{4pt}{6pt}
 Epsilon & Zeta  & Eta     & Theta \\
\specialrule{0.8pt,blue3}{3pt}{2pt}
 Iota    & Kappa & Lambda  & Mu    \\
\bottomrule
\end{booktabs}
\end{demohigh}

At last, there is also an \verb!\addlinespace! command, with an alternative
name \verb!\addrowspace!.
You can specify the size of vertical space to be added in its optional
argument, and the default size is determinted by \verb!\defaultaddspace!
dimension, initially \verb!0.5em!.
This command adds one half of the space to \verb!belowsep! of previous row,
and the other half to \verb!abovesep! of current row.

\begin{demohigh}
\begin{booktabs}{row{2}={olive9}}
\toprule
 Alpha   & Beta  & Gamma   & Delta \\
\addlinespace
 Epsilon & Zeta  & Eta     & Theta \\
\addlinespace[1em]
 Iota    & Kappa & Lambda  & Mu    \\
\bottomrule
\end{booktabs}
\end{demohigh}

From version 2022A (2022-03-01), there is a \verb!longtabs! environment
for writing long \verb!booktabs! tables,
and a \verb!talltabs! environment for writing tall \verb!booktabs! tables.

\section{Library \texttt{counter}}

You need to load \verb!counter! library with \verb!\UseTblrLibrary{counter}!,
if you want to modify some LaTeX counters inside \verb!tabularray! tables.

\begin{demohigh}
\newcounter{mycnta}
\newcommand{\mycnta}{\stepcounter{mycnta}\arabic{mycnta}}
\begin{tblr}{hlines}
  \mycnta & \mycnta & \mycnta \\
  \mycnta & \mycnta & \mycnta \\
  \mycnta & \mycnta & \mycnta \\
\end{tblr}
\end{demohigh}

\section{Library \texttt{diagbox}}

When writing \verb!\UseTblrLibrary{diagbox}! in the preamble of the document,
\verb!tabularray! package loads \verb!diagbox! package,
and you can use \verb!\diagbox! and \verb!\diagboxthree! commands inside \verb!tblr! environment.

\begin{demohigh}
\begin{tblr}{hlines,vlines}
 \diagbox{Aa}{Pp} & Beta & Gamma \\
 Epsilon & Zeta  & Eta \\
 Iota    & Kappa & Lambda \\
\end{tblr}
\end{demohigh}

\begin{demohigh}
\begin{tblr}{hlines,vlines}
 \diagboxthree{Aa}{Pp}{Hh} & Beta & Gamma \\
 Epsilon & Zeta  & Eta \\
 Iota    & Kappa & Lambda \\
\end{tblr}
\end{demohigh}

You can also use \verb!\diagbox! and \verb!\diagboxthree! commands in math mode.
\nopagebreak
\begin{demohigh}
$\begin{tblr}{|c|cc|}
\hline
 \diagbox{X_1}{X_2} & 0 & 1 \\
\hline
  0 & 0.1 & 0.2 \\
  1 & 0.3 & 0.4 \\
\hline
\end{tblr}$
\end{demohigh}

\section{Library \texttt{functional}}

With \verb!\UseTblrLibrary{functional}! in the preamble of the document,
\verb!tabularray! will load \href{https://ctan.org/pkg/functional}{\texttt{functional}} package,
and define outer key \verb!evaluate! and inner key \verb!process!.
These two new keys are useful for doing functional programming inside tables.

\subsection{Outer key \texttt{evaluate} in action}

With outer key \verb!evaluate!, you can evaluate every occurrence of a specified protected function
(defined with \verb!\prgNewFunction!) and replace it with the return value before splitting the table body.

The first application of \verb!evaluate! key is for inputting files inside tables.
Assume you have two files \verb!test1.tmp! and \verb!test2.tmp! with the following contents:

\begin{filecontents*}[overwrite]{test1.tmp}
Some & Some \\
\end{filecontents*}
\begin{filecontents*}[overwrite]{test2.tmp}
Other & Other \\
\end{filecontents*}

\begin{codehigh}
\begin{filecontents*}[overwrite]{test1.tmp}
Some & Some \\
\end{filecontents*}
\end{codehigh}

\begin{codehigh}
\begin{filecontents*}[overwrite]{test2.tmp}
Other & Other \\
\end{filecontents*}
\end{codehigh}

Then you can input them with outer specification \verb!evaluate=\fileInput!.
The \verb!\fileInput! function is provided by \verb!functional! package.

\begin{demohigh}
\begin{tblr}[evaluate=\fileInput]{hlines}
  Row1 & 1 \\
  \fileInput{test1.tmp}
  Row3 & 3 \\
  \fileInput{test2.tmp}
  Row5 & 5 \\
\end{tblr}
\end{demohigh}

In general, you can define your functions which return parts of table contents,
and use \verb!evaluate! key to evaluate them inside tables.

\begin{demohigh}
\IgnoreSpacesOn
\prgNewFunction \someFunc {m} {
  \prgReturn {#1 & #1 \\}
}
\IgnoreSpacesOff
\begin{tblr}[evaluate=\someFunc]{hlines}
  Row1 & 1 \\
  \someFunc{Text}
  Row3 & 3 \\
  \someFunc{Text}
  Row5 & 5 \\
\end{tblr}
\end{demohigh}

\begin{demohigh}
\IgnoreSpacesOn
\prgNewFunction \otherFunc {} {
  \prgReturn {Other & Other \\}
}
\IgnoreSpacesOff
\begin{tblr}[evaluate=\otherFunc]{hlines}
  Row1 & 1 \\
  \otherFunc
  Row3 & 3 \\
  \otherFunc
  Row5 & 5 \\
\end{tblr}
\end{demohigh}

You can even generate the whole table with some function.

\begin{demohigh}
\IgnoreSpacesOn
\prgNewFunction \makeEmptyTable {mm}  {
  \tlSet \lTmpaTl {\intReplicate {\intEval{#2-1}} {&}}
  \tlPutRight \lTmpaTl {\\}
  \intReplicate {#1} {\tlUse \lTmpaTl}
}
\IgnoreSpacesOff
\begin{tblr}[evaluate=\makeEmptyTable]{hlines,vlines}
  \makeEmptyTable{3}{7}
\end{tblr}
\end{demohigh}

From version 2023A, you can evaluate all functions in the table body
with option \texttt{evaluate=all}.

\subsection{Inner key \texttt{process} in action}

With inner key \verb!process!, you can modify the contents and styles before the table is built.
Several public functions defined with \verb!\prgNewFuncton! are provided for you:

\begin{itemize}
  \item \verb!\cellGetText{<rownum>}{<colnum>}!
  \item \verb!\cellSetText{<rownum>}{<colnum>}{<text>}!
  \item \verb!\cellSetStyle{<rownum>}{<colnum>}{<style>}!
  \item \verb!\rowSetStyle{<rownum>}{<style>}!
  \item \verb!\columnSetStyle{<colnum>}{<style>}!
\end{itemize}

As the first example, let's calculate the sums of cells column by column:

\IgnoreSpacesOn
\prgNewFunction \funcSum {} {
  \intStepOneInline {1} {\arabic{colcount}} {
    \intZero \lTmpaInt
    \intStepOneInline {1} {\arabic{rowcount}-1} {
      \intAdd \lTmpaInt {\cellGetText {####1} {##1}}
    }
    \cellSetText {\expWhole{\arabic{rowcount}}} {##1} {\intUse\lTmpaInt}
  }
}
\IgnoreSpacesOff
\begin{codehigh}
\IgnoreSpacesOn
\prgNewFunction \funcSum {} {
  \intStepOneInline {1} {\arabic{colcount}} {
    \intZero \lTmpaInt
    \intStepOneInline {1} {\arabic{rowcount}-1} {
      \intAdd \lTmpaInt {\cellGetText {####1} {##1}}
    }
    \cellSetText {\expWhole{\arabic{rowcount}}} {##1} {\intUse\lTmpaInt}
  }
}
\IgnoreSpacesOff
\end{codehigh}

\begin{demohigh}
\begin{tblr}{colspec={rrr},process=\funcSum}
\hline
  1 & 2 & 3 \\
  4 & 5 & 6 \\
  7 & 8 & 9 \\
\hline
    &   &   \\
\hline
\end{tblr}
\end{demohigh}

Now, let's set background colors of cells depending on their contents:

\IgnoreSpacesOn
\prgNewFunction \funcColor {} {
  \intStepOneInline {1} {\arabic{rowcount}} {
    \intStepOneInline {1} {\arabic{colcount}} {
      \intSet \lTmpaInt {\cellGetText {##1} {####1}}
      \intCompareTF {\lTmpaInt} > {0}
        {\cellSetStyle {##1} {####1} {bg=purple8}}
        {\cellSetStyle {##1} {####1} {bg=olive8}}
    }
  }
}
\IgnoreSpacesOff
\begin{codehigh}
\IgnoreSpacesOn
\prgNewFunction \funcColor {} {
  \intStepOneInline {1} {\arabic{rowcount}} {
    \intStepOneInline {1} {\arabic{colcount}} {
      \intSet \lTmpaInt {\cellGetText {##1} {####1}}
      \intCompareTF {\lTmpaInt} > {0}
        {\cellSetStyle {##1} {####1} {bg=purple8}}
        {\cellSetStyle {##1} {####1} {bg=olive8}}
    }
  }
}
\IgnoreSpacesOff
\end{codehigh}

\begin{demohigh}
\begin{tblr}{hlines,vlines,cells={r,$},process=\funcColor}
  -1 &  2 &  3 \\
   4 &  5 & -6 \\
   7 & -8 &  9 \\
\end{tblr}
\end{demohigh}

We can also use color series of \verb!xcolor! package to color table rows:

\definecolor{lightb}{RGB}{217,224,250}
\definecolorseries{tblrow}{rgb}{last}{lightb}{white}
\resetcolorseries[3]{tblrow}
\IgnoreSpacesOn
\prgNewFunction \funcSeries {} {
  \intStepOneInline {1} {\arabic{rowcount}} {
    \tlSet \lTmpaTl {\intMathMod {##1-1} {3}}
    \rowSetStyle {##1} {\expWhole{bg=tblrow!![\lTmpaTl]}}
  }
}
\IgnoreSpacesOff
\begin{codehigh}
\definecolor{lightb}{RGB}{217,224,250}
\definecolorseries{tblrow}{rgb}{last}{lightb}{white}
\resetcolorseries[3]{tblrow}
\IgnoreSpacesOn
\prgNewFunction \funcSeries {} {
  \intStepOneInline {1} {\arabic{rowcount}} {
    \tlSet \lTmpaTl {\intMathMod {##1-1} {3}}
    \rowSetStyle {##1} {\expWhole{bg=tblrow!![\lTmpaTl]}}
  }
}
\IgnoreSpacesOff
\end{codehigh}

\begin{demohigh}
\begin{tblr}{hlines,process=\funcSeries}
  Row1 & 1 \\
  Row2 & 2 \\
  Row3 & 3 \\
  Row4 & 4 \\
  Row5 & 5 \\
  Row6 & 6 \\
\end{tblr}
\end{demohigh}

\section{Library \texttt{hook}}

This library is experimental, please see\\
\url{https://github.com/lvjr/tabularray/wiki/HooksAndVariables}.

\section{Library \texttt{html}}

This library is experimental, please see\\
\url{https://github.com/lvjr/tabularray/wiki/HooksAndVariables}.

\section{Library \texttt{keyvalue}}

With \verb!\UseTblrLibrary{keyvalue}! in the preamble of the document,
\verb!tabularray! will load \href{https://ctan.org/pkg/functional}{\texttt{functional}} package,
so as to evaluate every function (defined with \verb!\prgNewFunction!) within inner specifications,
replacing it with its return value, before parsing the key-value pairs. Here is an example:

\begin{demohigh}
\begin{tblr}{
  row{2} = {bg=\intIfOddTF{\value{page}}{\prgReturn{red7}}{\prgReturn{blue7}}}
}
  Alpha   & Beta  & Gamma  \\
  Epsilon & Zeta  & Eta    \\
  Iota    & Kappa & Lambda \\
\end{tblr}
\end{demohigh}

Note that this library will conflict with inner key \verb!process! provided by \verb!functional! library.

\section{Library \texttt{nameref}}

From version 2022D, you can load \verb!nameref! library
to make \verb!\nameref! and \verb!longtblr! work together.

\section{Library \texttt{siunitx}}

When writing \verb!\UseTblrLibrary{siunitx}! in the preamble of the document,
\verb!tabularray! package loads \verb!siunitx! package,
and defines \verb!S! column as \verb!Q! column with \verb!si! key.

\begin{demohigh}
\begin{tblr}{
  hlines, vlines,
  colspec={S[table-format=3.2]S[table-format=3.2]}
}
 {{{Head}}} & {{{Head}}} \\
   111      &   111      \\
     2.1    &     2.2    \\
    33.11   &    33.22   \\
\end{tblr}
\end{demohigh}

\begin{demohigh}
\begin{tblr}{
  hlines, vlines,
  colspec={Q[si={table-format=3.2},c]Q[si={table-format=3.2},c]}
}
 {{{Head}}} & {{{Head}}} \\
   111      &   111      \\
     2.1    &     2.2    \\
    33.11   &    33.22   \\
\end{tblr}
\end{demohigh}

Note that you need to use \underline{\color{red3}triple} pairs of curly braces to guard non-numeric cells.
But it is cumbersome to enclose each cell with braces. From version 2022B (2022-06-01)
a new key \verb!guard! is provided for cells and rows. With \verb!guard! key the previous example
can be largely simplified.

\begin{demohigh}
\begin{tblr}{
  hlines, vlines,
  colspec={Q[si={table-format=3.2},c]Q[si={table-format=3.2},c]},
  row{1} = {guard}
}
   Head  & Head   \\
  111    & 111    \\
    2.1  &   2.2  \\
   33.11 &  33.22 \\
\end{tblr}
\end{demohigh}

Also you must use \verb!l!, \verb!c! or \verb!r! to set horizontal alignment for non-numeric cells:
\nopagebreak
\begin{demohigh}
\begin{tblr}{
  hlines, vlines, columns={6em},
  colspec={
    Q[si={table-format=3.2,table-number-alignment=left},l,blue7]
    Q[si={table-format=3.2,table-number-alignment=center},c,teal7]
    Q[si={table-format=3.2,table-number-alignment=right},r,purple7]
  },
  row{1} = {guard}
}
  Head  & Head   & Head   \\
 111    & 111    & 111    \\
   2.1  &   2.2  &   2.3  \\
  33.11 &  33.22 &  33.33 \\
\end{tblr}
\end{demohigh}

Both \verb!S! and \verb!s! columns are supported. In fact, These two columns have been defined as follows:
\begin{codehigh}
\NewColumnType{S}[1][]{Q[si={#1},c]}
\NewColumnType{s}[1][]{Q[si={#1},c,cmd=\TblrUnit]}
\end{codehigh}
You don't need to and are not allowed to define them again.

\section{Library \texttt{varwidth}}

To build a nice table, \verb!tabularray! need to measure the widths of cells.
By default, it uses \verb!\hbox! to measure the sizes.
This causes an error if a cell contains some vertical material, such as lists or display maths.

With \verb!\UseTblrLibrary{varwidth}! in the preamble of the document,
\verb!tabularray! will load \verb!varwidth! package,
and add a new inner specification \verb!measure! for tables.
After setting \verb!measure=vbox!, it will use \verb!\vbox! to measure cell widths.

\begin{demohigh}
\begin{tblr}{hlines,measure=vbox}
  Text Text Text Text Text Text Text
  \begin{itemize}
    \item List List List List List List
    \item List List List List List List List
  \end{itemize}
  Text Text Text Text Text Text Text \\
\end{tblr}
\end{demohigh}

From version 2022A (2022-03-01), you can remove extra space above and below lists,
by adding option \verb!stretch=-1!.
The following example also needs \verb!enumitem! package and its \verb!nosep! option:

{\centering\begin{tblr}{
  hlines,vlines,rowspec={Q[l,t]Q[l,b]},
  measure=vbox,stretch=-1,
}
  \begin{itemize}[nosep]
    \item List List List List List
    \item List List List List List List
  \end{itemize} & oooo \\
  \begin{itemize}[nosep]
    \item List List List List List
    \item List List List List List List
  \end{itemize} & gggg \\
\end{tblr}\par}

%% BUG: there is extra vertical space at the beginning of the first cell if I use demohigh
%\begin{demohigh}
\begin{codehigh}
\begin{tblr}{
  hlines,vlines,rowspec={Q[l,t]Q[l,b]},
  measure=vbox,stretch=-1,
}
  \begin{itemize}[nosep]
    \item List List List List List
    \item List List List List List List
  \end{itemize} & oooo \\
  \begin{itemize}[nosep]
    \item List List List List List
    \item List List List List List List
  \end{itemize} & gggg \\
\end{tblr}
\end{codehigh}
%\end{demohigh}

Note that option \verb!stretch=-1! also removes struts from cells, therefore it may not work well
in \verb!tabularray! environments with \verb!rowsep=0pt!, such as
\verb!booktabs!/\verb!longtabs!/\verb!talltabs! environments from \verb!booktabs! library.

\section{Library \texttt{zref}}

From version 2022D, you can load \verb!zref! library
to make \verb!\zref! and \verb!longtblr! work together.

\end{document}
